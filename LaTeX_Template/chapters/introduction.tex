\chapter{Einleitung}
\label{cha:Einleitung}
Public Adress Systeme oder kurz \textit{PA-Systeme} bieten die Möglichkeit, akustische Signale für eine Menschenmenge hörbar zu machen. Ihr Hauptzweck ist es, Musik oder Sprache möglichst Klanggetreu zu übertragen und eine für den Anwendungsbereich angemessene Lautstärke herzustellen. Dafür unterscheidet man die \textbf{ELAs (Elektroakustische Anlagen)}, welche als reine Sprach- und Ansagesysteme genutzt werden und die PA-Systeme, die zur Musikübertragung für Konzertbeschallung o.Ä. dienen. 

Die ersten Ideen der Beschallungstechnik entsprangen den zwanziger Jahren des 20. Jahrhunderts mit der \textit{Neumann-Flasche}, einem Kondensatormikrofon. Es wurde versucht, mit einfachen Mikrofon- und Lautsprecheranlagen Hörsäle zu beschallen. Mit der Verbreitung des Tonfilms, bestand der Zwang, leistungsfähigere Beschallungssysteme zu entwickeln, die später auch Verwendung in sicherheitsrelevanten Bereichen, wie dem Bahnbetrieb fanden.

Relevant für die Planung der Deutschen Bahn sind hier lediglich die Elektroakustischen Anlagen zur Sprach-, Ansagen- und Notfallansagenübertragung, z.B. im Falle eines Brandes. Im Folgenden wird der Begriff \textit{Beschallungsanlagen} für Elektroakustische Anlagen verwendet. Beschallungsanlagen werden vielfältig verteilt über die Bahninfrastruktur eingesetzt, beispielsweise zur Beschallung von Bahnsteigen und Gleisbereichen oder für Sprachalarmanlagen (SAA).
Beschallungsanlagen werden von Planungsingenieuren der DB geplant, d.h. sie müssen dem Stand der Technik, dazu gehören unterschiedlichste Normen und Richtlinien, entsprechen und sie müssen für den vorgesehenen Einsatz ausreichend dimensioniert sein. Diese Planer werden durch Arbeitshilfen bei der Planung unterstützt. Zu diesem Zwecke habe ich eine solche Arbeitshilfe zur \textit{Berechnung von Kabelquerschnitten für Lautsprecherkreise} erstellt.