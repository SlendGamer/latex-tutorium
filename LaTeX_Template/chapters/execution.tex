\chapter{Umsetzung}
\label{cha:Umsetzung}

\section{Realisierung der Arbeitshilfe in Excel}
\label{sec:Realisierung}
Das Excel-Tool hat zwei Arbeitsblätter, eines für die Eingabe aller für die Berechnung relevanten Daten und eines zur Ausgabe der Ergebnisse und Berechnungsformeln. Wie in der folgenden Grafik zu erkennen, ist der Eingabebereich auf dem ersten Tabellenblatt in den \textit{Soll}-Teil und den \textit{Ist}-Teil aufgeteilt.

\begin{figure}[H]
\centering
\includegraphics[width=1\linewidth]{images/Excel-Tool1}
\caption{Excel-Tool Oberfläche}
\label{fig:Excel-Tool1}
\end{figure}

Der Soll-Teil, beinhaltet alle Daten, die zur Berechnung des minimalen Kabelquerschnitts benötigt werden, dabei kann lediglich in die Zellen mit den Farben Gelb, Grün oder Orange ein Wert eingetragen werden. Alle anderen Zellen dienen zur Ausgabe der Zwischenergebnisse. Ein entsprechender Schutz des Arbeitsblattes mit einem Passwort verhindert eine unbefugte Veränderung der Berechnungsformeln in den Zwischenergebnis-Zellen. Da bei der DB KT stets TK-Kabel zum Einsatz kommen, die 2 Doppeladern besitzen, ist die Längenspezifikation in zwei verschiedene \textit{Lautsprecherkreise} unterteilt. Dabei wird, zur Berechnung des maximalen Spannungsabfalls am letzten Lautsprecher, geprüft, welche Leitung am längsten ist. Diese wird dann mit 2 multipliziert, um auch den Rückweg abzubilden. Der Kabelquerschnitt wird anhand der hergeleiteten Formel (\ref{eqn:flaeche3}) für die Mindestfläche bzw. (\ref{eqn:durchmesser}) den Mindestdurchmesser des Kabels berechnet. Weitere Zwischenergebnisse, wie die jeweiligen Leistungen und Widerstände dienen nur der Nachvollziehbarkeit. Für die Leistung der Lautspreceher ist die Gesamtzahl, der in beiden Lautsprecherkreisen vorkommenden Lautsprecher anzusetzen, damit der einzusetzende Verstärker ausreichend dimensioniert werden kann.

Im Ist-Teil trägt der Fachplaner den geplanten Kabeldurchmesser des A/B-Kabels ein. Ein grünes Feld bestätigt, dass das Kabel ausreichend groß dimensioniert ist, ein oranges Feld zeigt eine Diskrepanz mit den berechneten Werten an. Dasselbe wird beim Eintragen der geplanten Verstärkerleistung angezeigt. Wenn beides ausreichend dimensioniert ist, gibt ein Text darunter dies wieder. Weiterhin wird aus dem eingetragenen Kabeldurchmesser, die Querschnittsfläche berechnet, auf dessen Grundlage der Widerstand und die anfallende Leistung erfasst wird.

Auf der rechten Seite ist ebenfalls eine Legende, mit allen dargestellten Größen zu finden.

\begin{figure}[H]
\centering
\includegraphics[width=1\linewidth]{images/Legende}
\caption{Legende}
\label{fig:Legende}
\end{figure}

Darunter sind zwei Grafiken angeordnet, die die Verschaltung der Lautsprecher spezifizieren.

\begin{figure}[H]
\centering
\includegraphics[width=1\linewidth]{images/Grafik1}
\caption{Erläuterungsgrafik mit Längebezeichnung und Verschaltung}
\label{fig:Grafik1}
\end{figure} 

\begin{figure}[H]
\centering
\includegraphics[width=0.9\linewidth]{images/Grafik2}
\caption{Erläuterungsgrafik mit Übertrager-Schema}
\label{fig:Grafik2}
\end{figure}