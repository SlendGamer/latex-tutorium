% ----------------------------- PRÄAMBEL ----------------------------------------
\documentclass[
12pt,						% Schriftgröße
a4paper,					% Papierformat
oneside, 					% einseitiges (oneside) oder zweiseitiges (twoside) Dokument
listof=totoc, 				% Tabellen- und Abbildungsverzeichnis ins Inhaltsverzeichnis
bibliography=totoc,			% Literaturverzeichnis ins Inhaltsverzeichnis aufnehmen
titlepage, 					% Titlepage-Umgebung statt \maketitle
headsepline, 				% horizontale Linie unter Kolumnentitel
%abstracton,				% Überschrift beim Abstract einschalten, Abstract muss dazu in {abstract}-Umgebung stehen
DIV=12,						% Satzspiegeleinstellung, 12 ist Standard bei KOMA-Script
%BCOR=6mm,					% Bindekorrektur, die den Seitenspiegel um 6mm nach rechts verschiebt,
%cleardoublepage=empty,	% Stil einer leeren eingefügten Seite bei Kapitelwechsel
parskip,					% Absatzabstand bei Absatzwechsel einfügen
ngerman
]{scrbook}

\usepackage{scrhack}			
\usepackage[utf8]{inputenc} 	% ermöglicht die direkte Eingabe von Umlauten
\usepackage[T1]{fontenc} 		% Ausgabe aller zeichen in einer T1-Codierung (wichtig für die Ausgabe von Umlauten!)
\usepackage{babel} 				% deutsche Trennungsregeln und Übersetzung der festcodierten Überschriften (chapter -> kapitel, usw.)
\setlength{\parindent}{0ex} 	% bei neuem Abschnitt nicht einrücken

% Einstellung der Leitlinien
\usepackage[onehalfspacing]{setspace}

% ---------------------------------------------------------------------------------------------------------------------------------------
% Folgende Einstellungen sind bei größeren Arbeiten mit viel Text zu empfehlen.
% Hierbei oben DIV=16 einstellen und Zeile \usepackage[onehalfspacing]{setspace} auskommentieren.
% \linespread{1.2}\selectfont     % Zeilenabstand erhöhen - größere Werte als 1.2 nicht verwenden!!
% Ende Einstellung große Arbeiten mit viel Text.
% ---------------------------------------------------------------------------------------------------------------------------------------

\usepackage{siunitx}		% Vereinfacht die Eingabe von Einheiten in Formeln
\sisetup{
	number-unit-product = \;,
	inter-unit-product = \:,
	exponent-product = \cdot,
	output-decimal-marker = {,}
}

\usepackage{graphicx}  	% Einbinden von Grafiken erlauben
\usepackage{float}			% Ermöglicht "Floating-Objects" im Dokument, also Objekte, die frei auf der Seite platziert werden können
\usepackage{svg}			% Ermöglicht Einbindung von SVG-Vektordateien

\usepackage[
format=hang,		% Formatierungseinstellungen von Unter- / Überschriften
font=normal,
labelfont=bf,
justification=RaggedRight,
singlelinecheck=true,
aboveskip=1mm
]{caption}

\usepackage[
backend=biber, 		% Hilfsprogramm "biber" beim Compilieren nutzen (statt "biblatex" oder "bibtex")
style=alphabetic, 	% Zitierstil (siehe Dokumentation!!!)
natbib=true,		% Bereitstellen von natbib-kompatiblen Zitierkommandos
hyperref=true, 		% hyperref-Paket verwenden, um Links zu erstellen
]{biblatex}
\addbibresource{literature/literature.bib} 	% Einbinden der bib-Datei. Endung .bib unbedingt ergänzen

% Folgende Zeilen auskommentieren, falls runde Klammern und ein "vgl." bei Zitaten erscheinen sollen.
%\makeatletter
%\renewcommand{\@cite}[2]{(vgl. {#1\if@tempswa , #2\fi})} 
%\renewcommand{\@biblabel}[1]{(#1)}
%\makeatother

\usepackage{pdfpages}			% Ermöglicht Einführung von bestehenden PDF Seiten (z.B. Datenblätter, Zeichnungen, ...)

\usepackage{enumitem}			% Vielseitiges Package für Aufzählungen
\usepackage{tabto}				% Ermöglicht Verwendung von Tabstops (zur Einrückung von Text)
	
\usepackage{amsmath}			% Beinhaltet Ergänzungen für Formeln
\usepackage{textcomp} 			% Ermöglicht Einsatz von Sonderzeichen (z.B. Euro-Zeichen, ...)
\usepackage{eurosym}			% bessere Darstellung Euro-Symbol mit \euro
\newcommand*\diff{\mathop{}\!\mathrm{d}}				% Differentialzeichen
\newcommand*\Diff[1]{\mathop{}\!\mathrm{d^#1}} 		% Differentialzeichen höherer Ableitung
\newcommand*\jj{\mathop{}\!\mathrm{j}}					% Komplexe Zahl j (in E-Tech gebräuchlich)

\usepackage[					% Einstellungen für Referenzierung (bzw. Fußnoten)
hyperfootnotes=false			% im pfd-Output werden Fußnoten nicht verlinkt
]{hyperref}

\usepackage{makeidx}			% Paket zur Erstellung des Indexverzeichnisses (Alphabetisch Sortierte Stichwortverzeichnis)
\usepackage[intoc]{nomencl} 	% zur Erstellung des Abkürzungsberzeichnisses

\usepackage[					% Einstellungen für Fußnoten
bottom,							% Ausrichtung  = unten
multiple,						% Trennung mehrerer Fußnoten durch einen Separator
hang,
marginal
]{footmisc}

\usepackage{calc}				% Ermöglicht die Berechnung von Längen (z.B.: 0.8\linewidth)

\usepackage{xcolor} 			% Ermöglicht Verwendung von Farben in nahezu allen Farbmodellen

% Benutzerdefinierte Farben erzeugen
\definecolor{mygreen}{rgb}{0,0.6,0}
\definecolor{mygray}{rgb}{0.5,0.5,0.5}
\definecolor{mymauve}{rgb}{0.58,0,0.82}

\usepackage{listings}			% Darstellung von Quellcode mit den Umgebungen {lstlisting}, \lstinline und \lstinputlisting
\lstset{literate=					% Ermöglicht Umlaute innerhalb von Listing-Umgebungen
	{Ö}{{\"O}}1
	{Ä}{{\"A}}1
	{Ü}{{\"U}}1
	{ß}{{\ss}}1
	{ü}{{\"u}}1
	{ä}{{\"a}}1
	{ö}{{\"o}}1
}

\lstset{ %
	backgroundcolor=\color{white},   % choose the background color; you must add \usepackage{color} or \usepackage{xcolor}; should come as last argument
	basicstyle=\footnotesize,        % the size of the fonts that are used for the code
	breakatwhitespace=false,         % sets if automatic breaks should only happen at whitespace
	breaklines=true,                 % sets automatic line breaking
	captionpos=t,                    % sets the caption-position to (b) bottom or (t) top
	commentstyle=\color{mygreen},    % comment style
	deletekeywords={...},            % if you want to delete keywords from the given language
	escapeinside={\%*}{*)},          % if you want to add LaTeX within your code
	escapeinside={(*@}{@*)},
	extendedchars=true,              % lets you use non-ASCII characters; for 8-bits encodings only, does not work with UTF-8
	frame=none,	                   	% "single" adds a frame around the code; "none"
	keepspaces=true,                 % keeps spaces in text, useful for keeping indentation of code (possibly needs columns=flexible)
	keywordstyle=\color{blue},       % keyword style
	language=[LaTeX]TeX,             % the language of the code
	morekeywords={*,nomenclature},   % if you want to add more keywords to the set
	numbers=left,                    % where to put the line-numbers; possible values are (none, left, right)
	numbersep=5pt,                   % how far the line-numbers are from the code
	numberstyle=\tiny\color{mygray}, % the style that is used for the line-numbers
	rulecolor=\color{black},         % if not set, the frame-color may be changed on line-breaks within not-black text (e.g. comments (green here))
	showspaces=false,                % show spaces everywhere adding particular underscores; it overrides 'showstringspaces'
	showstringspaces=false,          % underline spaces within strings only
	showtabs=false,                  % show tabs within strings adding particular underscores
	stepnumber=1,                    % the step between two line-numbers. If it's 1, each line will be numbered
	stringstyle=\color{mymauve},     % string literal style
	tabsize=2,	                   % sets default tabsize to 2 spaces
	title=\lstname                   % show the filename of files included with \lstinputlisting; also try caption instead of title
}

% -------------------- Verzeichnisse erstellen -------------------- 
\makeindex				% Indexverzeichnis erstellen
\makenomenclature		% Abkürzungsverzeichnis erstellen

% -----------------------------------------------------------------------------------------------------------------
% Zum Aktualisieren des Abkürzungsverzeichnisses (Nomenklatur) bitte auf der Kommandozeile folgenden Befehl aufrufen :
% makeindex <Dateiname>.nlo -s nomencl.ist -o <Dateiname>.nls
% Oder besser: Kann in TexStudio unter Tools-Benutzer als Shortlink angelegt werden
% Konfiguration unter: Optionen-Erzeugen-Benutzerbefehle: makeindex -s nomencl.ist -t %.nlg -o %.nls %.nlo
% -----------------------------------------------------------------------------------------------------------------

% Hier die persönlichen Daten eingeben:
% Die Befehle können als Variablen betrachtet werden, die im Titelblatt eingesetzt werden
% Vereinzelt können hier auch passend für die Arbeit Felder leer gelassen werden

\newcommand{\titel}{Titel der wissenschaftlichen Arbeit}
\newcommand{\untertitel}{Untertitel (optional)}
\newcommand{\arbeit}{Projektbericht / Studienarbeit / Laborbericht / etc.}
\newcommand{\studiengang}{Studiengang}
\newcommand{\studienrichtung}{Studienrichtung}
\newcommand{\studienschwerpunkt}{Studienschwerpunkt}
\newcommand{\autor}{Name des Autors}
\newcommand{\matrikelnr}{Matrikelnummer}
\newcommand{\kurs}{Kursbezeichnung}
\newcommand{\firma}{Duales Partnerunternehmen}
\newcommand{\abgabe}{Datum der spätesten Abgabe}
\newcommand{\betreuerdhbw}{Betreuer der DHBW}
\newcommand{\betreuerfirma}{Betreuer des Partnerunternehmens}
\newcommand{\jahr}{20xx}			% für Angabe im Copyright-Vermerk der Titelseite

% Folgende Zeilen definieren Abkürzungen, um Befehle schneller eingeben zu können
% \newcommand{\ua}{\mbox{u.\,a.\ }} % Bei Bedarf ausklammern
% \newcommand{\zB}{\mbox{z.\,B.\ }} % Bei Bedarf ausklammern
\newcommand{\bs}{$\backslash$}

% Zur Einbindung von Tikz und PGF-Plot-Grafiken
\usepackage{pgfplots}
\usepackage{pgfplotstable}
\pgfplotsset{compat=newest,width=0.6\linewidth}
\usepgfplotslibrary{smithchart} 	% Einbindung von Smith-Diagrammen (HF-Technik) möglich
\usepackage{tikz}					% Tikz sollte nach Listings Pakete geladen werden.
\usetikzlibrary{arrows}

\hyphenation{Schrift-ar-ten} % Korrekte Worttrennung

% -------------------------------------------------------------------------------------------
%                     BEGINN DES DOKUMENTENINHALTS 
% -------------------------------------------------------------------------------------------
\begin{document}
\let\texteuro\euro					% Eingabe \texteuro, € oder \euro erzeugt gleiches Ergebnis
\setcounter{secnumdepth}{3}		% maximale Nummerierungstiefe fürs Inhaltsverzeichnis 
\setcounter{tocdepth}{3}
\sffamily							% für die Titelei wird serifenlose Schrift verwendet

% ----------------------------------- TITELEI -------------------------------------------------
\include{pages/titelseite} 				% erzeugt die Titelseite
\pagenumbering{roman}					% kleine, römische Seitenzahlen für Titelei
\include{pages/erklaerung} 			% Einbinden der eidestattlichen Erklärung
\chapter*{Kurzfassung} %*-Variante sorgt dafür, das Abstract nicht im Inhaltsverzeichnis auftaucht
\label{cha:abstract_ger}
Im Rahmen der ersten und zweiten Praxisphase wurde ein erster Einblick in das Geschäftsgeschehen möglich. So ergab es sich, die Tätigkeit eines Projektmanagers und Planungsingenieurs im Ansatz kennenzulernen. 

Ein Großteil der ersten Praxisphase wurde von mir damit verbracht, ein Excel-Tool zur automatischen Erstellung eines Planverzeichnisses nach betriebsinternen Vorgaben zu entwickeln, womit interne Geschäftsprozesse vereinfacht und beschleunigt werden sollen. Dieses Tool sollte Teil einer weitergehenden Software-Umgebung zur Vereinfachung, Beschleunigung und Vereinheitlichung interner Geschäftsprozesse werden.

Auf Grundlage der dadurch angeeigneten Erfahrungen und Fähigkeiten in der Entwicklung von Excel-Arbeitshilfen bot sich die Möglichkeit, ein Planungstool für Beschallungsanlagen zu entwickeln. Das Tool soll den Fachplaner bei der Berechnung von Leitungsquerschnitten einzelner Lautsprecherkreise der 100-Volt-Technik unterstützen. Ziel ist es, den Planungsprozess zu vereinfachen und zu beschleunigen.

\chapter*{Abstract}
\label{cha:abstract_eng}
The first and second practice semester gave a first impression on the business activities and duties. Because of the right circumstances, it was possible for me to get to know the basic activities of a project manager and a planning engineer.

A considerable part of my first semester was spent by me developing an Excel tool for automatically creating a plan directory with internal corporate specifications, the purpose of this being to simplify and accelerate internal business processes. Additionaly, this tool was planned to be part of a wider software tool environment for simplification, acceleration and unifying purposes of planning processes.

Based on the collected experience and skills by developing such a tool, the opportunity to create a planning tool for public adress systems arose. The tool is designed to help the specialised planning engineer with calculating wire line cross-sections of single speaker circuits in the subject of 100-volt-technology. The goal is to simplify and accelerate the planning process.

\cleardoublepage
   			% Einbinden des Abstracts

\tableofcontents						% Erzeugen des Inhaltsverzeichnisses
\cleardoublepage

% --------------------------------------------------------------------------------------------
%                    		Inhalt der Arbeit
%---------------------------------------------------------------------------------------------
\pagenumbering{arabic}					% arabische Seitenzahlen für den Hauptteil
\rmfamily

\include{chapters/einleitung}
\include{chapters/grundlagen}
\include{chapters/konzeptentwurf}
\include{chapters/umsetzung}
%\include{chapters/verifikation}
\include{chapters/zusammenfassung}

% ---- Literaturverzeichnis ----------
\interlinepenalty 10000				% Verhindert einen Umbruch mitten in Literatureinträgen
\nocite{*}								% Literaturverzeichnis ohne Zitation einblenden
\printbibliography						% Erstellen des Literaturverzeichnisses

% -----Ausgabe aller Verzeichnisse ---
\setlength{\parskip}{0.5\baselineskip}
%\renewcommand{\indexname}{Sachwortverzeichnis}
%\printindex												% Erzeugen des alphabetisch geordneten Indexverzeichnises
%\addcontentsline{toc}{chapter}{\indexname}

\input{pages/abkuerzungen}				% Datei mit allgemeinen Abkürzungen laden
\renewcommand{\nomname}{Verzeichnis verwendeter Formelzeichen und Abkürzungen}
\setlength{\nomlabelwidth}{.20\hsize}
\renewcommand{\nomlabel}[1]{#1 \dotfill}
\setlength{\nomitemsep}{-\parsep}
\printnomenclature						% Erzeugen des Abkürzungsverzeichnises, siehe auch Inhalt der Datei pages/abkuerzungen.tex
\cleardoublepage

%\renewcommand{\glossaryname}{Glossar} %Glossar erzeugen
%\printglossaries
%\cleardoublepage

%\listoffigures 							% Erzeugen des Abbildungsverzeichnisses 
%\cleardoublepage

%\listoftables 							% Erzeugen des Tabellenverzeichnisses
%\cleardoublepage

% -----Anhang ------------------------
%\appendix
%\clearpage
%\pagenumbering{Roman}					% große, römische Seitenzahlen für Anhang, falls gewünscht
%\include{chapter/anhang}
%\include{chapter/anhang_vorlagen}		% Zeile auskommentieren bei finalem Dokument!

\end{document}
